\documentclass{beamer}

%\usetheme{AnnArbor}
%\usetheme{Antibes}
%\usetheme{Bergen}
%\usetheme{Berkeley}
%\usetheme{Berlin}
%\usetheme{Boadilla}
%\usetheme{boxes}
%\usetheme{CambridgeUS}
%\usetheme{Copenhagen}
%\usetheme{Darmstadt}
%\usetheme{default}
%\usetheme{Frankfurt}
%\usetheme{Goettingen}
%\usetheme{Hannover}
%\usetheme{Ilmenau}
%\usetheme{JuanLesPins}
%\usetheme{Luebeck}
\usetheme{Madrid}
%\usetheme{Malmoe}
%\usetheme{Marburg}
%\usetheme{Montpellier}
%\usetheme{PaloAlto}
%\usetheme{Pittsburgh}
%\usetheme{Rochester}
%\usetheme{Singapore}
%\usetheme{Szeged}
%\usetheme{Warsaw}

\title{Zorn's Lemma}

\author{Vincent Chung}

\institute[University of Hawaii at Manoa]
{
  \inst{}%
  Math 480\\
  Professor DeMeo}

% Let's get started
\begin{document}

\begin{frame}
  \titlepage
\end{frame}

\begin{frame}{Outline}
  \tableofcontents
  % You might wish to add the option [pausesections]
\end{frame}

% Section and subsections will appear in the presentation overview
% and table of contents.
\section{Intuition}

\subsection{ZFC Set Theory}
\subsection{Definitions}

\begin{frame}{ZFC Set Theory}
  \begin{itemize}
  \item {
    Zermelo-Frankel Set Theory, with the inclusion of the Axiom of Choice
  }
  \item {
    Foundation of axiomatic set theory and mathematics
  }
  \end{itemize}
\end{frame}

\begin{frame}{Axiom of Choice}
    \begin{theorem}
    The Cartesian product of a collection of non-empty sets is non-empty (Zermelo, 1904)
    \end{theorem}
    \begin{itemize}
        \item {
        A bit controversial
        }
        \item{
        Sets need to be "well-defined"
        }
        \item{
        Creating sets out of "thin air"}
    \end{itemize}   
\end{frame}

\begin{frame}{Some Definitions}
    \begin{itemize}
        \item {
        A partially ordered set is a set that obeys the binary relation $\leq$ over some set $P$, denoted ($P$,$\leq$).
        }
        \item{
        A chain is a set of pairwise comparable elements in $P$}
    \end{itemize}
\end{frame}

\section{Zorn's Lemma}

\subsection{Some history}
\subsection{General sketch of the proof}

\begin{frame}{Zorn's Lemma}
  \begin{lemma}
  Let $P$ be a partially ordered set in which every chain has an upper bound. Then, $P$ has at least one maximal element (Kurotowski and Zorn, 1922)
  \end{lemma}
\end{frame}

\begin{frame}{Some History on Zorn's Lemma}
    \begin{itemize}
        \item First proved by Kazimierz Kurotowski in 1922
        \item Independently proven by Max Zorn in 1935, proposing Zorn's Lemma as a new axiom to set theory
        \item Replaced the well-ordering theorem
    \end{itemize}
\end{frame}

\begin{frame}{General Overview of Proof}
    \begin{itemize}
        \item {Requires the Axiom of Choice}
    \end{itemize}
    \begin{itemize}
        \item {Defining a choice function on a set of ordinals}
    \end{itemize}
    \begin{itemize}
        \item {Using transfinite recursion to derive some contradiction about maximality}
    \end{itemize}
\end{frame}

\begin{frame}{Using Zorn's Lemma}
  \begin{itemize}
      \item {Form a collection $\Lambda$ of desired object}
      \item Given a collection of the desired objects, the union of which will create a chain
      \item By Zorn's Lemma, we can find that a maximal element exists within this collection
  \end{itemize}
\end{frame}

\begin{frame}{Zorn's Lemma $\Rightarrow$ Axiom of Choice}
    \begin{itemize}
        \item An equivalent form of AC: For any relation $R$, there is a function $F \subseteq R$ with dom F = dom R
        \item We define a collection of functions $\Lambda$ = $\{$f \subseteq R \mid f \text {is a function}\}$
    \end{itemize}
\end{frame}

\begin{frame}{Application of Zorn's Lemma}
    \begin{itemize}
        \item Equivalent forms of Zorn's Lemma
        \begin{itemize}
            \item Axiom of Choice
            \item Well-Ordering Principal
            \item Hausdorff Maximal Principal
        \end{itemize}
        \item Zorn's Lemma has applications in other fields of mathematics
    \end{itemize}
\end{frame}

\begin{frame}{Zorn's Lemma in Mathematics}
    \begin{itemize}
        \item Hahn-Banach Theorem (functional analysis)
        \item Tychonoff's Theorem (topology)
        \item Every vector space has a Hamel basis (linear algebra)
        \item Every commutative ring has a proper maximal ideal (abstract algebra)
    \end{itemize}
  
\end{frame}

\section*{Summary}

\begin{frame}{Conclusion}
  \begin{itemize}
  \item
    Accepting the Axiom of Choice results in Zorn's Lemma
  \item
    Zorn's Lemma implies Axiom of Choice
  \item
    Zorn's Lemma has a lot of applicaitons in various fields of mathematics
  \end{itemize}

\end{frame}


\end{document}


